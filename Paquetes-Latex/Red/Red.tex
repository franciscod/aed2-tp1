\documentclass[10pt, a4paper]{article}
\usepackage[paper=a4paper, left=1.5cm, right=1.5cm, bottom=1.5cm, top=3.5cm]{geometry}
\usepackage[latin1]{inputenc}
\usepackage[T1]{fontenc}
\usepackage[spanish]{babel}
\usepackage{indentfirst}
\usepackage{fancyhdr}
\usepackage{latexsym}
\usepackage{lastpage}
\usepackage{aed2-symb,aed2-itef,aed2-tad}
\usepackage[colorlinks=true, linkcolor=blue]{hyperref}
\usepackage{calc}

\newcommand{\f}[1]{\text{#1}}
\renewcommand{\paratodo}[2]{$\forall~#2$: #1}

\sloppy



\hypersetup{%
 % Para que el PDF se abra a p�gina completa.
 pdfstartview= {FitH \hypercalcbp{\paperheight-\topmargin-1in-\headheight}},
 pdfauthor={C�tedra de Algoritmos y Estructuras de Datos II - DC - UBA},
 pdfkeywords={TADs b�sicos},
 pdfsubject={Tipos abstractos de datos b�sicos}
}

\parskip=5pt % 10pt es el tama�o de fuente

% Pongo en 0 la distancia extra entre �temes.
\let\olditemize\itemize
\def\itemize{\olditemize\itemsep=0pt}

% Acomodo fancyhdr.
\pagestyle{fancy}
\thispagestyle{fancy}
\addtolength{\headheight}{1pt}
\lhead{Algoritmos y Estructuras de Datos II}
\rhead{$1^{\mathrm{er}}$ cuatrimestre de 2012}
\cfoot{\thepage /\pageref{LastPage}}
\renewcommand{\footrulewidth}{0.4pt}

\author{Algoritmos y Estructuras de Datos II, DC, UBA.}
\date{}
\title{Tipos abstractos de datos b�sicos}

\begin{document}

	

	\begin{tad}{\tadNombre{DCNet}}
		\tadGeneros{dcNet}
		\tadIgualdadObservacional{d}{d'}{dcNet}{$ (topo(d) \igobs topo(d')) \land  ((\forall p: paquetes)\newline(paquete?(d,p) \igobs paquete?(d',p))
			\impluego( recorridoPaquete(d,p) \igobs recorridoPaquete(d',p) ) ) \land
			((\forall p: pc)(p \in pcs(topo(d)) \land p \in pcs(topo(d')) \impluego (paquetesEnEspera(d,p) \igobs paquetesEnEspera(d',p)  )$}
		\tadGeneradores
		\tadOperacion{crearRed}{topo}{dcNet}{}
		\tadOperacion{seg}{dcNet}{dcNet}{}
		\tadOperacion{mandarPaquete}{dcNet/dcn,pc/p1,pc/p2,paquete}{dcNet}
		{($p_1$ $\in$ pcs(topo($dcn$)) $\land$  $p_2$ $\in$ pcs(topo($dcn$))) $\yluego$ \\ conectadas?(topo($dcn$) , $p_1$ , $p_2$)}

		\tadObservadores
		\tadOperacion{recorridoPaquete}{dcNet/dcn,paquete/p}{secu((ip,interface)))}{paquete?($dcn$,$p$)}
		\tadOperacion{paquetesEnEspera}{dcNet/dcn,pc/p}{conj(paquete)}{$p$ $\in$ pcs(topo($dcn$))}
		\tadOperacion{paquete?}{dcNet,paquete}{bool}{}
		\tadOperacion{topo}{dcNet}{topologia}{}

		\tadOtrasOperaciones
		\tadOperacion{maxPaquetesMandados}{dcNet}{pc}{}
		\tadAxiomas[\paratodo{paquete}{p,p'},\paratodo{pc}{c,c'},\paratodo{dcNet}{dcn},\paratodo{topologia}{t}]
		
		\tadAxioma{topo(crearRed(t))}{t}
		\tadAxioma{topo(seg(dcn))}{topo(dcn)}
		\tadAxioma{topo(mandarPaquete(dcn,c,c',p))}{topo(dcn)}



	\end{tad}

\end{document}
