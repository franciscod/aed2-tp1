\documentclass[10pt, a4paper]{article}
\usepackage[paper=a4paper, left=1.5cm, right=1.5cm, bottom=1.5cm, top=3.5cm]{geometry}
\usepackage[latin1]{inputenc}
\usepackage[spanish]{babel}
\usepackage[T1]{fontenc}
\usepackage{indentfirst}
\usepackage{fancyhdr}
\usepackage{latexsym}
\usepackage{lastpage}
\usepackage{aed2-symb,aed2-itef,aed2-tad,caratula}
\usepackage[colorlinks=true, linkcolor=blue]{hyperref}
\usepackage{calc}

\newcommand{\f}[1]{\text{#1}}
\renewcommand{\paratodo}[2]{$\forall~#2$: #1}

\sloppy



\hypersetup{%
 % Para que el PDF se abra a página completa.
 pdfstartview= {FitH \hypercalcbp{\paperheight-\topmargin-1in-\headheight}},
 pdfauthor={Cátedra de Algoritmos y Estructuras de Datos II - DC - UBA},
 pdfkeywords={TADs básicos},
 pdfsubject={Tipos abstractos de datos básicos}
}

\parskip=5pt % 10pt es el tamaño de fuente

% Pongo en 0 la distancia extra entre ítemes.
\let\olditemize\itemize
\def\itemize{\olditemize\itemsep=0pt}

% Acomodo fancyhdr.
\pagestyle{fancy}
\thispagestyle{fancy}
\addtolength{\headheight}{1pt}
\lhead{Algoritmos y Estructuras de Datos II}
\rhead{$1^{\mathrm{er}}$ cuatrimestre de 2012}
\cfoot{\thepage /\pageref{LastPage}}
\renewcommand{\footrulewidth}{0.4pt}

% Encabezado
\lhead{Algoritmos y Estructuras de Datos I}
\rhead{Grupo 12}
% Pie de pagina
\renewcommand{\footrulewidth}{0.4pt}
\lfoot{Facultad de Ciencias Exactas y Naturales}
\rfoot{Universidad de Buenos Aires}

\begin{document}

% Datos de caratula
\materia{Algoritmos y Estructuras de Datos II}
\titulo{Trabajo Pr\'actico I}
%\subtitulo{}
\grupo{Grupo: 12}

\integrante{Demartino, Francisco}{348/14}{demartino.francisco@gmail.com}
\integrante{Paz, Maximiliano Le\'on}{XXX/XX}{asdm4xileon@gmail.com}
\integrante{Mena, Manuel}{313/14}{manuelmena1993@gmail.com}
\integrante{Pondal, Iv\'an}{078/14}{ivan.pondal@gmail.com}

\maketitle

\newpage

% Para crear un indice
%\tableofcontents

% Forzar salto de pagina
\clearpage

% Pueden modularizar el documento incluyendo otros .tex
% \include{observaciones}

% \section{Observaciones}
%
% 	\begin{enumerate}
% 		\item un item
% 		\item otro item
% 	\end{enumerate}

% Otro salto de pagina
% \newpage

% aca van los tads




\end{document}
