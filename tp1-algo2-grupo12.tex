\documentclass[10pt, a4paper, spanish]{article}
\usepackage[paper=a4paper, left=1.5cm, right=1.5cm, bottom=1.5cm, top=3.5cm]{geometry}
\usepackage[spanish]{babel}
\selectlanguage{spanish}
\usepackage[utf8]{inputenc}
\usepackage[T1]{fontenc}
\usepackage{indentfirst}
\usepackage{fancyhdr}
\usepackage{latexsym}
\usepackage{lastpage}
\usepackage{aed2-symb,aed2-itef,aed2-tad,caratula}
\usepackage[colorlinks=true, linkcolor=blue]{hyperref}
\usepackage{calc}

\newcommand{\f}[1]{\text{#1}}
\renewcommand{\paratodo}[2]{$\forall~#2$: #1}

\sloppy

\hypersetup{%
 % Para que el PDF se abra a página completa.
 pdfstartview= {FitH \hypercalcbp{\paperheight-\topmargin-1in-\headheight}},
 pdfauthor={Grupo 12 - 1c2015 - Algoritmos y Estructuras de Datos II - DC - UBA},
 pdfsubject={TP1}
}

\parskip=5pt % 5pt es el tamaño de fuente

% Pongo en 0 la distancia extra entre ítemes.
\let\olditemize\itemize
\def\itemize{\olditemize\itemsep=0pt}

% Acomodo fancyhdr.
\pagestyle{fancy}
\thispagestyle{fancy}
\addtolength{\headheight}{1pt}
\lhead{Algoritmos y Estructuras de Datos II}
\rhead{$1^{\mathrm{er}}$ cuatrimestre de 2015}
\cfoot{\thepage /\pageref{LastPage}}
\renewcommand{\footrulewidth}{0.4pt}

% Encabezado
\lhead{Algoritmos y Estructuras de Datos II}
\rhead{Grupo 12}
% Pie de pagina
\renewcommand{\footrulewidth}{0.4pt}
\lfoot{Facultad de Ciencias Exactas y Naturales}
\rfoot{Universidad de Buenos Aires}

\begin{document}

% Datos de caratula
\materia{Algoritmos y Estructuras de Datos II}
\titulo{Trabajo Práctico I}
%\subtitulo{}
\grupo{Grupo: 12}

\integrante{Pondal, Iván}{078/14}{ivan.pondal@gmail.com}
\integrante{Paz, Maximiliano León}{251/14}{m4xileon@gmail.com}
\integrante{Mena, Manuel}{313/14}{manuelmena1993@gmail.com}
\integrante{Demartino, Francisco}{348/14}{demartino.francisco@gmail.com}

\maketitle
\newpage

% Para crear un indice
%\tableofcontents

\clearpage

\section{TADs Auxiliares}
\par \textbf{TAD} pc, ifz, id, ipOrigen, ipDestino, prioridad, ifzOrigen, ifzDestino \textbf{ES} nat
\par \textbf{TAD} paquete \textbf{ES} tupla(id, ipOrigen, ipDestino, prioridad)
\par \textbf{TAD} segmento \textbf{ES} tupla(ipOrigen, ifzOrigen, ipDestino, ifzDestino)

\section{TAD \tadNombre{DCNet}}
\begin{tad}{\tadNombre{DCNet}}
	\tadGeneros{dcnet}

	\tadAlinearFunciones{compuQueMasEnvio}{dcnet/dcn,pc/p1,pc/p2,paquete}
	\tadIgualdadObservacional{d}{d'}{dcnet}{$ (topo(d) \igobs topo(d')) \land \newline
		((\forall p: pc)(p \in pcs(topo(d)) \land   p \in pcs(topo(d')) \impluego  \newline
		(buffer(d,p) \igobs buffer(d',p) \land \newline
		\#paquetesEnviados(d,p) \igobs \#paquetesEnviados(d',p) )\land \newline
		((\forall p: paquetes)((\exists  c: pc )(c \in pcs(topo(d') \land \newline 
		c \in pcs(topo(d') ) \yluego(p \in buffer(d,c) \land \newline
		p \in buffer(d',c) ) )\impluego \newline
		( recorridoPaquete(d,p) \igobs recorridoPaquete(d',p) ) ) $}
	\tadGeneradores
	\tadOperacion{CrearRed}{topo}{dcnet}{}
	\tadOperacion{Seg}{dcnet}{dcnet}{}
	\tadOperacion{PaquetePendiente}{dcnet/dcn,pc/p1,pc/p2,paquete}{dcnet}
	{$(p_1 \in pcs(topo(dcn)) \land  p_2 \in pcs(topo(dcn))) \yluego  conectadas?(topo(dcn) , p_1 , p_2)$}

	\tadObservadores
	\tadOperacion{recorridoPaquete}{dcnet/dcn,paquete/p}{secu((ip,interface)))}{$(\exists  c: pc )(c \in pcs(topo(dcn)) \yluego (p \in buffer(dcn,c))$}
	\tadOperacion{buffer}{dcnet/dcn,pc/p}{conj(paquete)}{$p$ $\in$ pcs(topo($dcn$))}
	\tadOperacion{$\#$paquetesEnviados}{dcnet/dcn,pc/p}{nat}{$p$ $\in$ pcs(topo($dcn$))}

	\tadOperacion{topo}{dcnet}{topologia}{}

	\tadOtrasOperaciones
	\tadOperacion{paqueteEnTransito?}{dcnet,paquete}{bool}{}
	\tadOperacion{perteneceBuffers?}{paquete,buffers}{bool}{}
	
	\tadOperacion{compuQueMasEnvio}{dcnet}{pc}{}
	\tadOperacion{auxMaxPaquetes}{dcnet,conj(pc)}{pc)}{}
	
	\tadOperacion{pasoSeg}{topo,buffers,buffers}{buffers}{}
	\tadOperacion{regresion}{topo,buffers,secu(buffers)}{buffers}{}
	\tadOperacion{generarHistoria}{dcnet,diccionario(pc,conj(paquete))}{secu(buffers)}{}
	\tadOperacion{auxDefinir}{buffers,pc,conj(paquete),conj(paquete)}{buffers}{}
	\tadOperacion{auxBorrar}{buffers,pc,conj(paquete),conj(paquete)}{buffers}{}
	\tadOperacion{transacion}{topo,buffers,conj(pc)}{buffers}{}
	\tadOperacion{envio}{topo,buffers,ip,conj(paquete)}{buffers}{}
	\tadOperacion{nuevosPaquetes}{buffers,buffers}{buffers}{}
	\tadOperacion{damePaquete}{conj(paquete)}{paquete}{}
	\tadOperacion{pasarA}{topologia,pc,pc}{pc}{}


	\tadAlinearAxiomas{$\#$paquetesEnviados(paquetePendiente(dcn,o,d,p),c)}

	\tadAxiomas[\paratodo{paquete}{p,p'},\paratodo{pc}{c,c'},\paratodo{dcnet}{dcn},\paratodo{topologia}{t}]

	\tadAxioma{topo(crearRed(t))}{t}
	\tadAxioma{topo(seg(dcn))}{topo(dcn)}
	\tadAxioma{topo(paquetePendiente(dcn,c,c',p))}{topo(dcn)}

	\tadAxioma{$\#$paquetesEnviados(crearRed(t),c)}{0}
	\tadAxioma{$\#$paquetesEnviados(seg(dcn),c)}{$\#$paquetesEnviados(dcn)}
	\tadAxioma{$\#$paquetesEnviados(paquetePendiente(dcn,o,d,p),c)}{\IF $c = o$ THEN $\#paquetesEnviados(dcn,c) + 1$ ELSE $\#paquetesEnviados(dcn,c)$ FI}

	\tadAxioma{buffer(dcn,c)}{obtener(c,regresion(topo(dcn),vacio,generarHistoria(dcn,vacio)))}

	\tadAxioma{compuQueMasEnvio(dcn)}{auxMaxPaquetes(dcn,pcs(topo(dcn)))}
	\tadAxioma{auxMaxPaquetes(dcn,cs)}{\IF $\emptyset?(sinUno(cs))$ THEN 
											$dameUno(cs)$ 
										ELSE { \IF $\#paquetesEnviados(dcn,dameUno(cs)<$
													$\#paquetesEnviados(dcn,auxMaxPaquetes$
													$(dcn,sinUno(cs))) $ THEN
													$auxMaxPaquetes(dcn,sinUno(cs))$
												ELSE
													$dameUno(cs)$
												FI}
										FI}

	\tadAxioma{paqueteEnTransito?(dcn,p)}{perteneceBuffers?(p,regresion(topo(dcn),vacio,\newline
											generarHistoria(dcn,vacio)))}
	\tadAxioma{perteneceBuffers?(p,bs)}{\IF $\emptyset?(claves(bs))$ THEN 
											$false$ 
										ELSE { \IF $p \in obtener(dameUno(claves(bs)),bs)$ THEN
													$true$
												ELSE 
													$perteneceBuffers?(p,borrar(dameUno(claves(bs)),bs))$
												FI
											}	
										FI}



	\tadAxioma{generarHistoria(crearRed(t),bs)}{bs \puntito <>}
	\tadAxioma{generarHistoria(seg(dcn),bs)}{ bs \puntito generarHistoria(dcn,$vacío$) }
	\tadAxioma{generarHistoria(paquetePendiente(dcn,o,d,p),bs)}{ \IF $def?(c,bs)$ THEN 
																	$generarHistoria(dcn,definir(c,n \cup obtener(o,bs),bs))$
																ELSE 
																	$generarHistoria(dcn,definir(c,n))$ 
																FI}

	\tadAxioma{auxBorrar(bs,c,b,p)}{\IF $\emptyset?(p - \{ b \})$ THEN $borrar(c,n)$ 
										ELSE $borrar(c,bs) \ \ definir(c,p - \{ b \},bs)$  FI 
	}

	\tadAxioma{regresion(t,bs,cbs)}{\IF $vacia?(fin(cbs))$ THEN $ pasoSeg(bs,t,prim(cbs))$ ELSE $ regresion(t,pasoSeg(bs,t,prim(cbs)),fin(cbs))$ FI}
	\tadAxioma{pasoSeg(t,bs,nbs)}{nuevosPaquetes(transacion(t,bs,claves(bs)) ,nbs)}
	\tadAxioma{transacion(t,bs,cp)}{\IF $\emptyset?(cp)$ THEN 
												$bs$ 
											ELSE 
												$transacion(t,envio(t,bs,dameUno(cp)),$
												$sinUno(cp))$ 
											FI}
	\tadAxioma{pasarA(t,o,d)}{$prim(caminoMin(t,o,d))$}												
	\tadAxioma{envio(t,bs,ip,cp)}{\IF $\emptyset?(damePaquete(cp))$ THEN
								$bs$
								ELSE{ \IF $pasarA(t,ip,destino(damePaquete(cp))) = destino(damePaquete(cp)))$ THEN
											$envio(t,quitarPaquete(bs,ip),ip,cp -{damePaquete(cp)}))$
									    ELSE
											$envio(t,quitarPaquete(pasarPaquete(bs,ip,damePaquete(cp)),ip)$
											$,ip,cp -{damePaquete(b)} ))$
										FI}
								FI}

	\tadAxioma{nuevosPaquetes(bs,nbs)}{\IF $\emptyset?(claves(nbs))$ THEN 
										 $bs$  
										ELSE  
										 	$nuevosPaquetes(auxDefinir(bs,dameUno(claves(nbs),obtener$ 
										 	$(dameUno(claves(nbs),nbs),obtener(dameUno$  
										 	$(claves(nbs),bs))),sinUno(nbs))$	
										FI}

	\noindent TAD buffers es diccionario(pc,conj(paquete)) 
\end{tad}


\clearpage

\section{TAD \tadNombre{Topología}}
\begin{tadx}{\tadNombre{Topología}}{
	\noindent Este TAD modela cómo se conectan las computadoras.
	Las IP son únicas entre compus de la topología.
	Las compus tienen interfaces numeradas con los naturales de manera consecutiva
	(todas funcionan perfecto y todo eso, el DC las cuida y mantiene como corresponde).
	}
	\tadGeneros{topologia}
	% TODO: exporta?
	\tadIgualdadObservacional{t}{t'}{topo}{
		(compus($t$) \igobs compus($t'$)) \yluego \\
		(($\forall$ p : pc) (p $\in$ compus($t$) \impluego ( \\
			\-  $\qquad$ (cablesEn($t$, $p$) \igobs cablesEn($t'$, $p$)) $ \land $ \\
			\- $\quad$ (\#interfaces($t$, $p$) \igobs \#interfaces($t'$, $p$)) \\
		)))
	}
	\tadAlinearFunciones{interfacesOcupadasDe}{topologia, conj(nat), nat, conj(nat), secu(nat)	}
	\tadGeneradores
	\tadOperacion{NuevaTopo}{}{topologia}{}
	\tadOperacion{Compu}{topologia, pc/ip, nat}{topologia}{$\neg(ip \in compus(t))$}
	\tadOperacion{Cable}{topologia, pc/ipA, ifz/ifA, pc/ipB, ifz/ifB}{topologia}{
		$(ipA \in compus(t) \land ipB \in compus(t)) \yluego$ \\
		$(ifA < \#interfaces(t, ipA)) \land$ \\
		$(ifB < \#interfaces(t, ipB)) \land$ \\
		$\neg(ifA \in interfacesOcupadasDe(t, ipA)) \land$ \\
		$\neg(ifB \in interfacesOcupadasDe(t, ipB)) \land$ \\
		$\neg(ipA \in vecinas(t, ipB))$
	}
	\tadObservadores

	\tadOperacion{compus}{topologia}{conj(pc)}{}
	\tadOperacion{cablesEn}{topologia/t, pc/ip}{conj(tupla(pc, ifz))}{$ip \in compus(t)$}
	\tadOperacion{\#interfaces}{topologia/t, pc/ip}{nat}{$ip \in compus(t)$}

	\tadOtrasOperaciones

	\tadOperacion{vecinas}{topologia/t, pc/ip}{conj(pc)}{$ip \in compus(t)$}
	\tadOperacion{interfacesOcupadasDe}{topologia/t, pc/ip}{conj(ifz)}{$ip \in compus(t)$}
	\tadOperacion{conectadas?}{topologia/t, pc/ipA, pc/ipB}{bool}{$ipA \in compus(t) $ $\land$ $ipB \in compus(t)$}
	\tadOperacion{darInterfazConectada}{{conj(tupla(pc, ifz))}/cablesA, pc/ipB}{ifz}{$ipB \in ips(cablesA)$}
	\tadOperacion{darSegmento}{topologia/t, pc/ipA, pc/ipB}{segmento}{$ipA \in compus(t) \yluego ipB \in vecinas(t, ipA)$}
	\tadOperacion{estáEnRuta?}{secu(segmento)/ruta, pc/ip}{bool}{}
	\tadOperacion{darSiguientePc}{secu(segmento)/ruta, pc/ip}{pc}{$est\acute{a}EnRuta?(ruta, ip)$}
	\tadOperacion{darCaminoMasCorto}{topologia/t, pc/ipA, pc/ipB}{secu(segmento)}{$ipA \in compus(t)$ $\land$ $ipB \in compus(t)$ $\yluego conectadas?(t, ipA, ipB)$}

	\tadAlinearFunciones{darRutasVecinas}{topologia, conj(pc), pc/ip, conj(pc), secu(segmento)}

	\tadOperacion{darRutas}{topologia, pc/ipA, pc/ipB, conj(pc), {secu(segmento)}}{conj(secu(segmento))}{$ipA \in compus(t) \land ipB \in compus(t)$}
	\tadOperacion{darRutasVecinas}{topologia/t, conj(pc)/vec, pc/ip, conj(pc), secu(segmento)}{conj(secu(segmento))}{$ip \in compus(t) \land vec \subseteq compus(t)$}
	\tadOperacion{longMenorSec}{conj(secu($\alpha$))/secus}{nat}{$\neg \emptyset?(secus)$}
	\tadOperacion{secusDeLongK}{conj(secu($\alpha$)), nat}{conj(secu($\alpha$))}{}
	\tadOperacion{ips}{conj({tupla(pc, ifz)})}{conj(pc)}{}
	\tadOperacion{interfaces}{conj({tupla(pc, ifz)})}{conj(ifz)}{}

	\tadAlinearAxiomas{cablesEn(Compu($t$, $ipNueva$, $cantIfaces$), $ip$)}

	\tadAxiomas

	\tadAxioma{compus(NuevaTopo)}{$\emptyset$}
	\tadAxioma{compus(Compu($t$, $ipNueva$, $cantIfaces$))}{Ag($ipNueva$, compus($t$))}
	\tadAxioma{compus(Cable($t$, $ipA$, $ifA$, $ipB$, $ifB$))}{compus($t$)}
	\tadAxioma{cablesEn(NuevaTopo, $ip$)}{$\emptyset$}
	\tadAxioma{cablesEn(Compu($t$, $ipNueva$, $cantIfaces$), $ip$)}{cablesEn($t$, $ip$)}
	\tadAxioma{cablesEn(Cable($t$, $ipA$, $ifA$, $ipB$, $ifB$), $ip$)}{
	\IF $ip$ = $ipA$ THEN Ag($\langle$ $ipB$, $ifA$ $\rangle$, $\emptyset$) ELSE $\emptyset$ FI $\cup$ \\
	\IF $ip$ = $ipB$ THEN Ag($\langle$ $ipA$, $ifB$ $\rangle$, $\emptyset$) ELSE $\emptyset$ FI $\cup$ \\
	cablesEn($t$, $ip$)
	}

	\tadAlinearAxiomas{\#interfaces(Compu($t$, $ipNueva$, $cantIfaces$), $ip$)}
	\tadAxioma{\#interfaces(NuevaTopo, $ip$)}{0}
	\tadAxioma{\#interfaces(Compu($t$, $ipNueva$, $cantIfaces$), $ip$)}{\IF $ip$ = $ipNueva$ THEN $cantIfaces$ ELSE \#interfaces($t$, $ip$) FI}
	\tadAxioma{\#interfaces(Cable($t$, $ipA$, $ifA$, $ipB$, $ifB$), $ip$)}{\#interfaces($t$, $ip$)}

	\tadAxioma{interfacesOcupadasDe($t$, $ip$)}{interfaces(cablesEn($t$, $ip$))}
	\tadAxioma{vecinas($t$, $ip$)}{ips(cablesEn($t$, $ip$))}
	\tadAxioma{conectadas?($t$, $ipA$, $ipB$)}{
		$\neg$ $\emptyset$?(darRutas($t$, $ipA$, $ipB$, $\emptyset$, <>))
	}
		\tadAxioma{darInterfazConectada($conjCablesIpA$, $ipB$)}{
			\IF $ipB$ = $\pi_1$(dameUno($conjCablesIpA$)) THEN $\pi_2$(dameUno($conjCablesIpA$)) ELSE darInterfazConectada(sinUno($conjCablesIpA$), $ipB$) FI
		}
		\tadAxioma{darSegmento($t$, $ipA$, $ipB$)}{
			$\langle$ $ipA$, darInterfazConectada(cablesEn($t$, $ipA$), $ipB$), \\ $ipB$, darInterfazConectada(cablesEn($t$, $ipB$), $ipA$) $\rangle$
		}
		\tadAxioma{estáEnRuta?($ruta$, $ip$)}{
					\IF vacía?($ruta$) THEN
						false
					ELSE{
						\IF ipOrigen(prim($ruta$)) = $ip$ THEN
							true
						ELSE
							estáEnRuta?(fin($rutas$), $ip$)
						FI
						}
					FI
		}
	\tadAxioma{darSiguientePc($ruta$, $ip$)}{
					\IF ipOrigen(prim($ruta$)) = $ip$ THEN
						ipDestino(prim($ruta$))
					ELSE
						darSiguientePc(fin($rutas$), $ip$)
					FI
			}

		\tadAxioma{darCaminoMasCorto($t$, $ipA$, $ipB$)}{dameUno(secusDeLongK(darRutas($t$, $ipA$, $ipB$, $\emptyset$, <>), longMenorSec(darRutas($t$, $ipA$, $ipB$, $\emptyset$, <>)))}

		\tadAlinearAxiomas{darRutas($t$, $ipA$, $ipB$, $rec$, $ruta$)}

	\tadAxioma{darRutas($t$, $ipA$, $ipB$, $rec$, $ruta$)}{
		\IF $ipB$ $\in$ vecinas($t$, $ipA$) THEN
			Ag($ruta$ \circulito darSegmento($t$, $ipA$, $ipB$) , $\emptyset$)
		ELSE{
			\IF $\emptyset$?(vecinas($t$, $ipA$) - $rec$) THEN
				$\emptyset$
			ELSE
				darRutas($t$, dameUno(vecinas($t$, $ipA$) - $rec$), \\ $ipB$, Ag($ipA$, $rec$),\\
								$ruta$ \circulito darSegmento($t$, $ipA$, dameUno(vecinas($t$, $ipA$) - $rec$))) $\cup$ \\
				darRutasVecinas($t$, sinUno(vecinas($t$, $ipA$) - $rec$), \\ $ipB$, Ag($ipA$, $rec$), \\
								$ruta$ \circulito darSegmento($t$, $ipA$, dameUno(vecinas($t$, $ipA$) - $rec$)))
			FI
		}
		FI
	}

	\tadAlinearAxiomas{darRutasVecinas($t$, $vecinas$, $ipB$, $rec$, $ruta$)}
	\tadAxioma{darRutasVecinas($t$, $vecinas$, $ipB$, $rec$, $ruta$)}{
		\IF $\emptyset$?($vecinas$) THEN
			$\emptyset$
		ELSE
			darRutas($t$, dameUno($vecinas$), $ipB$, $rec$, $ruta$) $\cup$ \\
			darRutasVecinas($t$, sinUno($vecinas$), $ipB$, $rec$, $ruta$)
		FI
	}

	\tadAxioma{darCaminoMasCorto($t$, $ipA$, $ipB$)}{
		dameUno(secusDeLongK(darRutas($t$, $ipA$, $ipB$, $\emptyset$, <>), \\
		longMenorSec(darRutas($t$, $ipA$, $ipB$, $\emptyset$, <>)))
	}

	\tadAxioma{secusDeLongK($secus$, $k$)}{
		\IF $\emptyset$?($secus$) THEN
			$\emptyset$
		ELSE{
			\IF long(dameUno($secus$)) = $k$ THEN
				dameUno($secus$) $\cup$ secusDeLongK(sinUno($secus$), $k$)
			ELSE
				secusDeLongK(sinUno($secus$), $k$)
			FI
		}
		FI
	}

	\tadAxioma{longMenorSec($secus$)}{
		\IF $\emptyset$?(sinUno($secus$)) THEN
			long(dameUno($secus$))
		ELSE
			min(long(dameUno($secus$)), \\
			longMenorSec(sinUno($secus$)))
		FI
	}

	\tadAxioma{ips($conjDuplas$)}{
		\IF $\emptyset$?($conjDuplas$) THEN
			$\emptyset$
		ELSE
			Ag($\pi_1$(dameUno($conjDuplas$)), \\
			ips(sinUno($conjDuplas$)))
		FI
	}

	\tadAxioma{interfaces($conjDuplas$)}{
		\IF $\emptyset$?($conjDuplas$) THEN
			$\emptyset$
		ELSE
			Ag($\pi_2$(dameUno($conjDuplas$)), \\
			interfaces(sinUno($conjDuplas$)))
		FI
	}

\end{tadx}


\end{document}
